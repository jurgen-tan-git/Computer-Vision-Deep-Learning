\documentclass{article}
\usepackage{amsmath}

\begin{document}

\section*{Problem 2.1: Expected Zero-One-Loss}
The expected zero-one-loss is given by:
\begin{align*}
E(x, y) & \sim P[I[f_0(x) \neq y]] \\
& = P[Y = +1] \cdot P[f_0(x) \neq +1 | Y = +1] \\
& \quad + P[Y = -1] \cdot P[f_0(x) \neq -1 | Y = -1] \\
& = p^+ \cdot 0.5 + (1 - p^+) \cdot 0.5 \\
& = 0.5
\end{align*}
So, the expected zero-one-loss for $f_0(x)$ is 0.5.

\section*{Problem 2.2: Probability of Zero Training Loss}
The probability of zero training loss with $N$ samples is given by:
\begin{align*}
P(\text{Zero training loss}) & = \left[P(X(1) < 0 | Y = +1)\right]^{N/2} \cdot \left[P(X(1) \geq 0 | Y = +1)\right]^{N/2} \\
& \quad \cdot \left[P(X(1) < 0 | Y = -1)\right]^{N/2} \cdot \left[P(X(1) \geq 0 | Y = -1)\right]^{N/2} \\
& = (0.5^{N/2}) \cdot (0.5^{N/2}) \cdot (0.5^{N/2}) \cdot (0.5^{N/2}) \\
& = 0.25^N
\end{align*}
So, the probability of zero training loss with $N$ samples is $0.25^N$.

\section*{Problem 2.3: Evolution of Probability with Increasing $N$}
As $N$ increases to infinity, the probability of zero training loss, $P(\text{Zero training loss})$, approaches zero. This is because as $N$ becomes very large, the probability of all the samples satisfying the conditions for zero training loss becomes extremely low, and it becomes increasingly unlikely to obtain zero training loss as $N$ goes to infinity.

\end{document}
